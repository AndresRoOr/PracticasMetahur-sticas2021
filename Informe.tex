\documentclass{article}
\title{Informe}
\usepackage[utf8]{inputenc}
\usepackage{algorithm}
\usepackage{algorithmic}
\usepackage{}
\author{David Díaz Jiménez, Andrés Rojas Ortega}

\begin{document}
	
	\maketitle
	
	\section{Definición y análisis del problema}
	
	Dado un conjunto N de tamaño n se pide encontrar un subconjunto M de tamaño m que maximice
	la función: 
	
	\[ d(s_i,M)=\sum_{s_j \in M} d(s_i,s_j)\]
	donde  $d(s_i,s_j)$ es la distancia del elemento $s_i$ al elemento $s_j$
	
	\subsection{Representación de la solución}
	
	Para representar la solución se ha optado por el uso de un vector de enteros, en el que él elemento contenido en cada posición se corresponde con un integrante de la solución.
	
	\subsection{Función objetivo}
	
	\[ d(s_i,M)=\sum_{s_j \in M} d(s_i,s_j)\]
	
	\section{Pseudocódigo}
	
	\subsection{Greedy}
	
	\subsubsection{Algoritmo principal}
	
		\begin{algorithm}
			\caption{Algoritmo Greedy}
			\begin{algorithmic}
				\STATE $solucion \leftarrow GeneraSolucionInicial(semilla)$
				\STATE $candidatos \leftarrow GeneraCandidatos()$
				\WHILE{! SolucionEncontrada()}
				\STATE $candidato \leftarrow Seleccion(candidatos)$
				\IF{Factible(candidato)}
				\STATE $solucion \leftarrow solucion \cup \{candidato\}$
				\STATE $sumaResultado \leftarrow sumaResultado + Coste(candidato)$
				\ENDIF
				\STATE $candidatos \leftarrow candidatos - \{ candidato \}$
				\ENDWHILE
			\end{algorithmic}
		\end{algorithm}

	\subsubsection{Generador de la solución inicial}
	
		\begin{algorithm}
			\caption{GeneraSolucionInicial(semilla)}
			\begin{algorithmic}
				\STATE $solucion \leftarrow \emptyset$
				\STATE $solucionInicial \leftarrow GeneraEnteroAlearorio(semilla)$
				\STATE $solucion \leftarrow solucion \cup \{solucionInicial\}$
				\RETURN solucion
			\end{algorithmic}
		\end{algorithm}

	\subsubsection{Generador del conjunto de candidatos}

	\begin{algorithm}
		\caption{GeneraCandidatos()}
		\begin{algorithmic}
			\STATE $candidatos \leftarrow \emptyset$
			\FORALL {$elemento \in matrizDatos$}
			\IF {$elemento \notin solucion$}
			\STATE $candidatos \leftarrow candidatos \cup \{elemento\}$
			\ENDIF
			\ENDFOR
			\RETURN candidatos
		\end{algorithmic}
	\end{algorithm}

	\subsubsection{Función solución}

	\begin{algorithm}
		\caption{SolucionEncontrada()}
		\begin{algorithmic}	
			\RETURN tamañoSolucion $<$ tamañoSolucionObjetivo 
		\end{algorithmic}
	\end{algorithm}

	\subsubsection{Función selección}

	\begin{algorithm}
		\caption{Seleccion(candidatos)}
		\begin{algorithmic}
			\STATE $max \leftarrow 0$
			\STATE $seleccionado \leftarrow \emptyset$
			\FORALL {$candidato \in candidatos$}
			\IF {Coste(candidato)$>$max}
			\STATE $max \leftarrow Coste(candidato)$
			\STATE $seleccionado \leftarrow candidato$ 
			\ENDIF
			\ENDFOR
			\RETURN seleccionado
		\end{algorithmic}
	\end{algorithm}

	\subsubsection{Función de factibilidad}

	\begin{algorithm}
		\caption{Factible(candidato)}
		\begin{algorithmic}
			\RETURN candidato != -1
		\end{algorithmic}
	\end{algorithm}

	
	\subsection{Búsqueda Local}
	
	\subsection{Búsqueda Tabú}
	
	
	\section{Experimentos y análisis de resultados}
	
	\subsection{Parámetros de los algoritmos}
	
	\subsubsection{Semillas}
	
	\subsection{Resultados obtenidos}
	
	\subsection{Análisis de los resultados}
	
	\subsection{Análisis de los algoritmos}
	

\end{document}